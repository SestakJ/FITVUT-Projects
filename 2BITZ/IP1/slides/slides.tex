\documentclass[xcolor=dvipsnames]{beamer}
\usepackage[utf8]{inputenc}
\usepackage[slovak]{babel}
\usepackage{graphics}
\usepackage{color}
\usepackage{amsfonts}
\usepackage{wrapfig}
\usepackage{tikz}
\usepackage[export]{adjustbox}

\usetheme{PaloAlto}
\useinnertheme{rounded}
\beamertemplatenavigationsymbolsempty
\setbeamertemplate{footline}[frame number]
\newcommand{\myuv}[1]{\quotedblbase #1\textquotedblleft}
 
 
%Information to be included in the title page:
\title{Abstrakcia nad konečnými automatmi v analýze programov}
\subtitle{Projektová prax 1} 
\author{Dávid Bolvanský}
\institute{Vysoké učení technické v Brně \\Fakulta informačních technologií}
 
 
 
\begin{document}
 
\frame{\titlepage}

\begin{frame}
\frametitle{Obsah}
\tableofcontents
\end{frame}

\section{Úvod}
 
\begin{frame}
\frametitle{Úvod}
\begin{itemize}
\item Široké využitie automatov v informačných technológiách
\item Formálna analýza programov, spracovanie textu
\item Rozhodovanie o prázdnosti jazyku automatu, ktorý je výsledkom operácií nad automatmi
\item Automatové algoritmy často založené na konštrukcii produktu
\item Kvadratická zložitosť produktu
\item Snaha o aproximácie
\item Naša problematika: aproximácia testu prázdnosti prieniku jazykov rozpoznávaných automatmi

\end{itemize}
\end{frame}

\section{Prienik jazykov}
 
\begin{frame}
\frametitle{Prienik jazykov automatov}
\begin{block}{Prázdnosť prieniku jazykov rozpoznávaných automatmi $A$ a $B$}
	\center{$L(A) \cap L(B) = \emptyset$}
\end{block}

\begin{block}{Test prázdnosti jazyka}
Test prázdnosti jazyka rozpoznávaného automatom analyzuje dostupnosť koncových stavov v produkte automatov od počiatočného stavu.
\end{block}


\begin{block}{Aproximácia testu prázdnosti prieniku}
Definujme $\Phi(A)$ ako aritmetickú formulu popisujúcu dlžky slov v jazyku rozpoznávaného automatom $A$.
	\center{$UNSAT(\Phi(A) \wedge \Phi(B)) \implies L(A) \cap L(B) = \emptyset$}

\end{block}
\end{frame}

\section{Aproximácia testu prázdnosti}

\begin{frame}
\frametitle{Postup aproximácie testu prázdnosti}
\begin{itemize}
\item Na začiatku máme nederministický konečný automat $A$.
\end{itemize}

\begin{center}
\begin{tikzpicture}[scale=0.2]
\tikzstyle{every node}+=[inner sep=0pt]
\draw [black] (12.4,-17.9) circle (3);
\draw (12.4,-17.9) node {$q_{0}$};
\draw [black] (24.8,-17.9) circle (3);
\draw (24.8,-17.9) node {$q_{1}$};
\draw [black] (40.6,-13.2) circle (3);
\draw (40.6,-13.2) node {$q_{2}$};
\draw [black] (40.6,-13.2) circle (2.4);
\draw [black] (40.6,-23.2) circle (3);
\draw (40.6,-23.2) node {$q_{3}$};
\draw [black] (40.6,-23.2) circle (2.4);
\draw [black] (39.277,-10.52) arc (234:-54:2.25);
\draw (40.6,-5.95) node [above] {$a$};
\fill [black] (41.92,-10.52) -- (42.8,-10.17) -- (41.99,-9.58);
\draw [black] (41.923,-25.88) arc (54:-234:2.25);
\draw (40.6,-30.45) node [below] {$b$};
\fill [black] (39.28,-25.88) -- (38.4,-26.23) -- (39.21,-26.82);
\draw [black] (27.68,-17.04) -- (37.72,-14.06);
\fill [black] (37.72,-14.06) -- (36.82,-13.8) -- (37.1,-14.76);
\draw (33.61,-16.11) node [below] {$a$};
\draw [black] (27.64,-18.85) -- (37.76,-22.25);
\fill [black] (37.76,-22.25) -- (37.16,-21.52) -- (36.84,-22.47);
\draw (31.72,-21.09) node [below] {$a$};
\draw [black] (4.6,-17.9) -- (9.4,-17.9);
\fill [black] (9.4,-17.9) -- (8.6,-17.4) -- (8.6,-18.4);
\draw [black] (15.4,-17.9) -- (21.8,-17.9);
\fill [black] (21.8,-17.9) -- (21,-17.4) -- (21,-18.4);
\draw (18.6,-18.4) node [below] {$b$};
\end{tikzpicture}
\end{center}

\end{frame}

\begin{frame}
\frametitle{Krok 1: Prevod na jednopísmenkový NKA}
\begin{itemize}
\item Následne ho prevedieme na jednopísmenkový nedeterministický konečný automat $A'$, ktorý ma iný jazyk, ale zároveň platí, že $\Phi(A) = \Phi(A')$.
\end{itemize}

\begin{center}
\begin{tikzpicture}[scale=0.2]
\tikzstyle{every node}+=[inner sep=0pt]
\draw [black] (12.4,-17.9) circle (3);
\draw (12.4,-17.9) node {$q_{0}$};
\draw [black] (24.8,-17.9) circle (3);
\draw (24.8,-17.9) node {$q_{1}$};
\draw [black] (40.6,-13.2) circle (3);
\draw (40.6,-13.2) node {$q_{2}$};
\draw [black] (40.6,-13.2) circle (2.4);
\draw [black] (40.6,-23.2) circle (3);
\draw (40.6,-23.2) node {$q_{3}$};
\draw [black] (40.6,-23.2) circle (2.4);
\draw [black] (39.277,-10.52) arc (234:-54:2.25);
\draw (40.6,-5.95) node [above] {$*$};
\fill [black] (41.92,-10.52) -- (42.8,-10.17) -- (41.99,-9.58);
\draw [black] (41.923,-25.88) arc (54:-234:2.25);
\draw (40.6,-30.45) node [below] {$*$};
\fill [black] (39.28,-25.88) -- (38.4,-26.23) -- (39.21,-26.82);
\draw [black] (27.68,-17.04) -- (37.72,-14.06);
\fill [black] (37.72,-14.06) -- (36.82,-13.8) -- (37.1,-14.76);
\draw (33.61,-16.11) node [below] {$*$};
\draw [black] (27.64,-18.85) -- (37.76,-22.25);
\fill [black] (37.76,-22.25) -- (37.16,-21.52) -- (36.84,-22.47);
\draw (31.72,-21.09) node [below] {$*$};
\draw [black] (4.6,-17.9) -- (9.4,-17.9);
\fill [black] (9.4,-17.9) -- (8.6,-17.4) -- (8.6,-18.4);
\draw [black] (15.4,-17.9) -- (21.8,-17.9);
\fill [black] (21.8,-17.9) -- (21,-17.4) -- (21,-18.4);
\draw (18.6,-18.4) node [below] {$*$};
\end{tikzpicture}
\end{center}

\end{frame}

\begin{frame}
\frametitle{Krok 2: Determinizácia jednopísmenkového NKA}
\begin{itemize}
\item Zdeterminizujeme jednopísmenkový NKA na jednopísmenkový derministický konečný automat $A''$.
\end{itemize}

\begin{center}
\begin{tikzpicture}[scale=0.2]
\tikzstyle{every node}+=[inner sep=0pt]
\draw [black] (12.4,-17.9) circle (3);
\draw (12.4,-17.9) node {$\{q_{0}\}$};
\draw [black] (24.8,-17.9) circle (3);
\draw (24.8,-17.9) node {$\{q_{1}\}$};
\draw [black] (38.2,-17.9) circle (3);
\draw (38.2,-17.9) node {\tiny{$\{q_{2}, q_{3}\}$}};
\draw [black] (38.2,-17.9) circle (2.4);
\draw [black] (27.8,-17.9) -- (35.2,-17.9);
\fill [black] (35.2,-17.9) -- (34.4,-17.4) -- (34.4,-18.4);
\draw (31.5,-18.4) node [below] {$*$};
\draw [black] (4.6,-17.9) -- (9.4,-17.9);
\fill [black] (9.4,-17.9) -- (8.6,-17.4) -- (8.6,-18.4);
\draw [black] (15.4,-17.9) -- (21.8,-17.9);
\fill [black] (21.8,-17.9) -- (21,-17.4) -- (21,-18.4);
\draw (18.6,-18.4) node [below] {$*$};
\draw [black] (36.877,-15.22) arc (234:-54:2.25);
\draw (38.2,-10.65) node [above] {$*$};
\fill [black] (39.52,-15.22) -- (40.4,-14.87) -- (39.59,-14.28);
\end{tikzpicture}
\end{center}

\end{frame}


\begin{frame}
\frametitle{Krok 3: Zistenie dĺžok slov}
\begin{itemize}
\item Každy jednopísmenkový deterministický konečný automat ma dve časti - rukoväť (handle) a laso (loop).

\begin{center}
\begin{tikzpicture}[scale=0.15]
\tikzstyle{every node}+=[inner sep=0pt]
\draw [black] (12.4,-17.9) circle (3);
\draw (12.4,-17.9) node {$q_{0}$};
\draw [black] (24.8,-17.9) circle (3);
\draw (24.8,-17.9) node {$q_{1}$};
\draw [black] (24.8,-17.9) circle (2.4);
\draw [black] (37.2,-17.9) circle (3);
\draw (37.2,-17.9) node {$q_{2}$};
\draw [black] (51.6,-17.9) circle (3);
\draw (51.6,-17.9) node {$q_{3}$};
\draw [black] (51.6,-17.9) circle (2.4);
\draw [black] (57.8,-8.9) circle (3);
\draw (57.8,-8.9) node {$q_{4}$};
\draw [black] (45.1,-8.9) circle (3);
\draw (45.1,-8.9) node {$q_{5}$};
\draw [black] (45.1,-8.9) circle (2.4);
\draw [black] (27.8,-17.9) -- (34.2,-17.9);
\fill [black] (34.2,-17.9) -- (33.4,-17.4) -- (33.4,-18.4);
\draw (31,-18.4) node [below] {$*$};
\draw [black] (4.6,-17.9) -- (9.4,-17.9);
\fill [black] (9.4,-17.9) -- (8.6,-17.4) -- (8.6,-18.4);
\draw [black] (15.4,-17.9) -- (21.8,-17.9);
\fill [black] (21.8,-17.9) -- (21,-17.4) -- (21,-18.4);
\draw (18.6,-18.4) node [below] {$*$};
\draw [black] (40.2,-17.9) -- (48.6,-17.9);
\fill [black] (48.6,-17.9) -- (47.8,-17.4) -- (47.8,-18.4);
\draw (44.4,-18.4) node [below] {$*$};
\draw [black] (58.236,-11.842) arc (-4.61158:-64.51347:6.588);
\fill [black] (58.24,-11.84) -- (57.67,-12.6) -- (58.67,-12.68);
\draw (57.7,-16.4) node [right] {$*$};
\draw [black] (47.821,-7.656) arc (107.46434:72.53566:12.091);
\fill [black] (47.82,-7.66) -- (48.73,-7.89) -- (48.43,-6.94);
\draw (51.45,-6.6) node [above] {$*$};
\draw [black] (48.864,-16.708) arc (-123.64616:-164.67853:8.51);
\fill [black] (48.86,-16.71) -- (48.48,-15.85) -- (47.92,-16.68);
\draw (46.09,-15.99) node [left] {$*$};
\end{tikzpicture}
\end{center}

\item Pre náš ukážkový automat platí, že stavy $q_{0}, q_{1}, q_{2}, q_{3}$ sú v rukoväti a stavy $q_{3}, q_{4}, q_{5}$ v lase.
\end{itemize}
\end{frame}

\begin{frame}
\frametitle{Krok 4: Zostavenie aritmetickej formule}

\begin{block}{Aritmetická formula popisujúca dĺžky slov pre koncové stavy automatu $A$}
$$
\Phi_q(A) =
  \begin{cases}
   |d| = v & \text{ak $q$ v rukoväti, $v > n$} \\
   |d| = v + k*l & \text{ak $q$ v lase, $v \geq n$}
  \end{cases}
$$
\begin{itemize}
\item n je dĺžka rukoväte
\item v je dľžka od počiatočného stavu po konečný stav
\item k je dľžka lasa
\item l je počet cyklení v lase
\end{itemize}
\end{block}

\begin{block}{Aritmetická formula popisujúca dĺžky slov jazyku rozpoznávaného automatom $A$}
\center{$\Phi(A) = \bigvee_{q \in F} \Phi_q(A)$}
\end{block}
\end{frame}

\begin{frame}
\frametitle{Krok 5: Overenie splniteľnosti aritmetickej formule}

\begin{block}{Súvis splniteľnosti aritmeckej formule s prázdnosťou prieniku jazykov}
Zostavíme aritmetické formule $\Phi(A)$ a $\Phi(B)$ a následne overíme splniteľnosť aritmetickej formule $\Phi(A) \wedge \Phi(B)$. Táto formula môže byť: \\
\begin{itemize}
\item splniteľná - nevieme rozhodnúť o prázdnosti či neprázdnosti prieniku ($\rightarrow$ použitie pokročilejších postupov aproximácie, Parikhove obrazy) \\
\item nesplniteľná - prienik jazykov rozpoznávaných automatmi je prázdny
\end{itemize}

\end{block}
\end{frame}

\section{Implementácia algoritmu}

\begin{frame}
\frametitle{Implementácia algoritmu na výpočet dĺžok slov}
\begin{itemize}
\item Implementácia algoritmu v Jave
\item Základné operácie s automatmi pomocou knižnice BRICS
\item Zvolený Alt-Ergo ako SMT solver na riešenie splniteľnosti aritmetickej formule $\Phi(A) \wedge  \Phi(B)$
\end{itemize}
\end{frame}


\section{Zhrnutie}

\begin{frame}
\frametitle{Zhrnutie}
\begin{itemize}
\item Problematika rozhodovania o prázdnosti prieniku jazykov rozpoznávanými automatmi
\item Jednoduchá aproximácia pomocou dĺžkok slov v jazyku automatu
\item Implementácia algoritmu na výpočet dĺžok slov
\item Do budúcna... prepojenie konštrukcie produktu s dĺžkami slov a Parikhových obrazov (Parikh images)
\end{itemize}
\end{frame}

\begin{frame}
\frametitle{Záver}
\center{\Huge{Ďakujem za pozornosť.}}
\end{frame}
 
\end{document}


